\documentclass[a4paper,11pt,oneside,onecolumn]{article}

\usepackage{a4wide}	%Blad wordt breeder benut
\usepackage{graphicx} %Afbeeldingen kunnen invoeren en opmaken
\usepackage{amsmath} %Geavanceerde wiskunde gebruiken
\usepackage[hang,small,bf]{caption} %Mooie bijschriften voor tabellen en figuren
\usepackage{url} %Mooie opmaak voor url
\usepackage[latin1]{inputenc} %Kan rechtstreeks met accenten enzo werken
\usepackage[Gray,squaren,thinqspace,thinspace]{SIunits} %Om grootheden en eenheden mooi te noteren
\usepackage{amssymb} %Wiskundige lettertypes
\usepackage{float} %Figuren mooi plaatsen
\usepackage[version=3]{mhchem} %Chemische formules
\usepackage{subfigure} %Figuren naast elkaar in zelfde omgeving

%\setlength{\parindent}{0pt} %Niet inspringen van nieuwe paragrafen
%\newcommand{\npar}{\par \vspace{2.3ex plus 0.3ex minus 0.3ex}} %Commando om nieuwe paragrafen te beginnen met witregel
\renewcommand{\vec}[1]{\mathbf{#1}} %Het \vec commando zet geen pijltje meer boven maar drukt vet

\begin{document}
\begin{titlepage}
		\begin{center}
			
			\vspace{5cm}			
			\begin{figure}[H]
				\begin{center}
					\includegraphics[width=4.5cm]{logo_ugent}
					\hspace{5cm}
					\includegraphics[width=3.75cm]{FEA_logo}
				\end{center}
			\end{figure}
			
			\vspace{2cm}
			\Large{Master of Electromechanical Engineering\\[2mm]Academic year 2014-2015}

			\vspace{3cm}
			\huge{\textsc{ICT \& Mechatronics\\ 
			Group D2 : Robot Arm}}

	
			
			\vspace{2cm}
			\large{Emanuel De Peuter\\ Quinten Martens\\ Frederik Ostyn\\ Wouter Staessens\\ Thibault Stepman\\ Dries T'Jampens}

			\end{center}	

			\vspace{2cm}

			\begin{large} \begin{tabbing}
			Supervisors: \= Dr. Ir. F. De Belie \\
			\> Prof. Dr. Ir. R. De Keyser \\
			\> Dr. Ir. C. Copot \\
			\> Dr. Ir. B. Van de Wiele
			\end{tabbing} \end{large}

\end{titlepage}

\pagenumbering{roman}

\tableofcontents


\newpage
\setcounter{page}{0}
\pagenumbering{arabic}

\section{Introduction}
\section{Task Description}

\section{Robot Kinematics}
In this part, the kinematics of a robot arm will be discussed. First, the Denavit-Hartenberg description will briefly be reviewed. Secondly, the calibration of the robot with the use of a MATLAB script will be presented. Thirdly, the range of the robot will be calculated.

\subsection{Mathematical framework}
A position and rotation of a link of the robot (or equivalently, the local robot coordinate system) as seen from a fixed global coordinate sytem can be represented by matrix $g$
\begin{equation}
g = \left(
\begin{matrix}
q & p \\
0 & 1
\end{matrix}
\right)
\end{equation}
where $p$ and $q$ represent respectively the coordinates of the origin of the robot coordinate system in the global system and the rotation matrix from global to robot frame.

If the robot has multiple links (which is the case here), the matrix $g_{end}$ representing the position and the orientation of the end effector can be build by multiplication of matrices
\begin{equation}
g_{end}=h_1h_2h_3h_4h_5
\label{directKin}
\end{equation}
where $h_i$ describes the position and rotation of link $i$ as seen from the coordinate system of link $i-1$ with $i=0$ the global (external) coordinate system, $i=1$ the base, $i=2$ the shoulder, $i=3$ the elbow, $i=4$ the wrist and $i=5$ the tip of the gripper (or pencil if applicable\footnote{In the Arduino program $h_5=h_{5b}h_{pen}$ with $h_5$ the matrix describing the gripper tip in the frame of the wrist and $h_{pen}$ the pencil tip in the frame of the gripper tip.}). The matrices $h_i$ were already given in the Arduino template and will therefore not be repeated here.

The above operation described in Eq. (\ref{directKin}) deals with the direct kinematics of the robot arm where the rotation of the servos is known and the matrix $g_{end}$ looked for. The inverse kinematics problem involves finding the correct servo angles if a coordinate and orientation of the end effector in the global system is given. This problem is far more difficult and only an approximate solution based on the Denavit-Hartenberg approach will be presented.

The inverse kinematics problem comes down to solving
\begin{equation}
b=Mx
\end{equation}
for $x$, namely $x=M^{-1}b$. Here, $x$ and $b$ are the components of the velocity matrix $\mu$, defined as
\begin{equation}
\mu = \left(
\begin{matrix}
0 & -\omega_z & \omega_y & v_x \\
\omega_z & 0 & -\omega_x & v_y \\
-\omega_y & \omega_x & 0 & v_z \\
0 & 0 & 0 & 0
\end{matrix}
\right)
\end{equation}
in different bases
\begin{equation}
\mu=\sum_{i=1}^5 b_i B_i=\sum_{i=1}^5 x_i X_i
\end{equation}
where $\omega_x$, $\omega_y$, $\omega_z$, $v_x$, $v_y$ and $v_z$ are the angular velocity and the velocity of the end effector arm in its own reference coordinate system\footnote{Remark that $\omega_z\equiv 0$ for this robot as it can turn its wrist up and down and clockwise/anticlockwise, but not in the $z-$direction.}. This matrix $\mu$ was calculated based on the property that $dg_{end}(t)/dt=g(t)\cdot \mu(t)$

In the above mathematical construction, $M$ is the transformation matrix between the bases $X$ and $B$, respectively associated with coordinates $x$ and $b$. The columns of $M$ contain the components of the base vectors $X_i$ as expressed in the base $B$. The base vectors $B_i$ are 
$$
B_1 = \left(
\begin{matrix}
0 & 0 & 0 & 0 \\
0 & 0 & -1 & 0 \\
0 & 1 & 0 & 0 \\
0 & 0 & 0 & 0
\end{matrix}
\right),
%
B_2 = \left(
\begin{matrix}
0 & 0 & 1 & 0 \\
0 & 0 & 0 & 0 \\
-1 & 0 & 0 & 0 \\
0 & 0 & 0 & 0
\end{matrix}
\right),
$$
$$
B_3 = \left(
\begin{matrix}
0 & 0 & 0 & 1 \\
0 & 0 & 0 & 0 \\
0 & 0 & 0 & 0 \\
0 & 0 & 0 & 0
\end{matrix}
\right),
B_4 = \left(
\begin{matrix}
0 & 0 & 0 & 0 \\
0 & 0 & 0 & 1 \\
0 & 0 & 0 & 0 \\
0 & 0 & 0 & 0
\end{matrix}
\right),
%
B_5 = \left(
\begin{matrix}
0 & 0 & 0 & 0 \\
0 & 0 & 0 & 0 \\
0 & 0 & 0 & 1 \\
0 & 0 & 0 & 0
\end{matrix}
\right)
$$
and the base vectors $X_i$ are
$$
X_1 = g_{end}^{-1}h_1s_1h_2h_3h_4h_{5b}h_{pen}
$$
$$
X_2 = g_{end}^{-1}h_1h_2s_2h_3h_4h_{5b}h_{pen}
$$
$$
X_3 = g_{end}^{-1}h_1s_1h_2h_3s_3h_4h_{5b}h_{pen}
$$
$$
X_4 = g_{end}^{-1}h_1s_1h_2h_3h_4s_4h_{5b}h_{pen}
$$
$$
X_5 = g_{end}^{-1}h_1s_1h_2h_3h_4h_{5b}s_5h_{pen}
$$
with
$$
s_1 = \left(
\begin{matrix}
0 & 1 & 0 & 0 \\
-1 & 0 & 0 & 0 \\
0 & 0 & 0 & 0 \\
0 & 0 & 0 & 0
\end{matrix}
\right),
%
s_2 = -s_3 = s_4\left(
\begin{matrix}
0 & 0 & 0 & 0 \\
0 & 0 & 1 & 0 \\
0 & -1 & 0 & 0 \\
0 & 0 & 0 & 0
\end{matrix}
\right),
%
s_5 = \left(
\begin{matrix}
0 & 0 & -1 & 0 \\
0 & 0 & 0 & 0 \\
1 & 0 & 0 & 0 \\
0 & 0 & 0 & 0
\end{matrix}
\right)
$$
As both bases are now defined, the coordinate transformation matrix $M$ can be calculated.

The only thing left to do is to find the components $b_i$ of $\mu$ in the base $B_i$. Recalling that 
\begin{equation}
dg_{end}(t)/dt=g(t)\cdot \mu(t)
\end{equation}
and assuming that in a certain step $dt$ the end effector should go from $g_{end}$ to $g_{des}$ with $g_{des}$ the desired state, the general solution in the case of $\mu$ being constant during the step can be found as being
\begin{equation}
g_{des} \equiv g(t+dt)=g(t)e^{\mu dt} \equiv g_{end}e^{\mu dt}
\label{dsE}
\end{equation}
where $e^{\mu dt}$ is a matrix exponential. As we know the base $B_i$, we know the components $b_i$ of $\mu$ if we know $\mu$. Eq. (\ref{dsE}) however is not trivial to solve. Therefore $e^{\mu dt}$ is approximated by restricting the number of terms in the Taylor expansion
\begin{equation}
e^{\mu dt} \approx e+\mu dt
\end{equation}
with $e$ the unit matrix. The solution for $\mu dt$ is\footnote{$\mu dt$ is called $dsE$ in the Arduino code.} then
\begin{equation}
dsE := \mu dt \approx g_{end}^{-1}g_{des}-e 
\end{equation}
From this result, the $b_i dt$ components can be calculated, which completes the mathematical frame work to find $du_i$ as
\begin{equation}
x dt= M^{-1} b dt \Leftrightarrow  du_i = \sum_j M^{-1}_{ij}b_j dt
\end{equation}

\subsection{Calibration of the robot}

In order to have a tool to calibrate the robot and to simulate how the presented mathematical framework describes the robot, a MATLAB script containing the direct and inverse algorithms was written. The direct kinematics part calculates and plots $g_{end}$ and the appropriate coordinate systems. The inverse kinematics part asks for a state $g_{des}$ and returns the servo angles that do the job.
\begin{figure}[h]
	\centering
		\includegraphics[width=\textwidth]{robotMatlab}
	\caption{Matlab simulation of the robot arm and the global and end effector coordinate system. This simulation was without the pencil; the end effector is the tip of the gripper.}
	\label{fig:robotMatlab}
\end{figure} 

\subsection{Range of the robot arm}

Dat met die cilinders... Wouter?
\section{Algorithm to go from point A to B}

\section{Sensor system design}

\section{Algorithm in order to avoid obstacles}

\section{Testing of the combined system}

\section{Conclusion}

\end{document}